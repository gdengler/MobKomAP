\section{Grundgedanken zu Mobilkommunkation}
\subsection{Ressourcenbedarf und -planung}
\begin{itemize}
\item Bandbreite
\begin{itemize}
\item Anzahl Teilnehmer
\end{itemize}
\item Zellengrösse
\begin{itemize}
\item kleine Zellen: viele Teilnehmer
\item grosse Zellen: schnell bewegende Teilnehmer
\end{itemize}
\end{itemize}

\subsection{zu haltenden Daten}
\begin{itemize}
\item Teilnehmer ID
\item Zielnummer abhängig von der Position
\begin{itemize}
\item Zelle
\item Location Area
\end{itemize}
\item Authentisierung
\end{itemize}

\subsection{Datenschutz}
\begin{itemize}
\item Teilnehmer muss Authentisiert werden
\item Netz muss Authentisiert werden
\end{itemize}
\subsection{Interoperabilität}
\begin{itemize}
\item 
\end{itemize}

\subsection{Accounting}
\begin{itemize}
\item Rechnungstelle
\begin{itemize}
\item Flatrate, best Effort
\item aktive Zeiten messen
\end{itemize}
\end{itemize}

\subsection{Operation}
\begin{itemize}
\item aktive Überwachung
\begin{itemize}
\item mit Personal aktiv Überwachen
\item Management Protokolle
\item dezentrale Wartungstrupps
\end{itemize}
\end{itemize}
\subsection{Generierung neuer Dienste}
\begin{itemize}
\item QoS
\item Netzauslastung lässt sich schwer voraussagen
\end{itemize}
\subsection{Akzeptanz bei Bevölterung}
\begin{itemize}
\item emotionsgeladene Diskussion auch unter Fachleute
\item schädlich oder nicht?
\begin{itemize}
\item unterschiedliche Aussagen
\item Tatsache: Es gibt keine Todesfälle die eindeutig auf Elektromagnetische Strahlung zurück zu führen sind
\end{itemize}
\item Umweltschäden durch Nutzerverhalten
\begin{itemize}
\item Recycling z.B. in Afrika nicht Umweltgerecht
\item sehr kurze Produktzyklen
\item höherer Schaden durch Nutzerverhalten als durch Strahlung
\end{itemize}
\item viele Dienste sind sehr viel schädlicher (z.B. Strassenverkehr, Rauchen, Hygiene in Spitälern)
\end{itemize}

\subsection{Anmeldung}
Bei der Anmeldung an das Netz muss die Mobilstation verchieden Probleme lösen. Welche sind das und wie können diese gelöst werden? \\
\begin{itemize}
\item Authentisierung des Benutzer
\item Authentisierung des Netzes
\item Anntene finden
\begin{itemize}
\item höhren und die Antenne mit stärksten Empfang wählen
\end{itemize}
\item Abstand zur Antenne unbekannt
\begin{itemize}
\item Paket kleiner Machen
\item nach Empfang erster Pakete kann die Entfernung eingeschätzt werden
\end{itemize}
\end{itemize}