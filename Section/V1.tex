\section{Einführung in die Mobilkommunikation}
\subsection{Multiplexing}
\subsubsection{Raum-Multiplex}
Raum wird in verschiedene Bereiche aufgeteilt und jedem Konkurrenten/Teilnehmer ein Bereich zugewisen, so dsas diese sich bei zeitgleicher Nutzung nicht gegenseiteig beeinflussen können. \\
\textbf{Beispiel :} Mobilfunk-Zellenaufteilung, Kabelbündel

\subsubsection{Zeit-Multiplex TDMA}
Das Medium der Kommunikation darf von jedem Konkurrenten exklusiv für eine bestimmte Periode genutzt werden (Zeitabschnitt wird zugewiesen, fest oder zufällig je nach Bedarf) \\
\textbf{Beispiel :} Festnetzte,GSM,IEEE 802,3 (CSMA/CD)

\subsubsection{Frequenz-Muliplex FDMA}
Jedem Konkurrenten um das MEdium wird eine exklusive Frequenz zugewiesen, auf der er dann zeitgleich mit den anderen senden kann.
\textbf{Beispiel :} Radio-/TV-Frequenz, GSM-Netz, Betriebsfunknetze,Glasfasernetze

\subsubsection{Code-Multiplex CDMA}
Zuordnung exklusiver Codes zu einzelnen Sendern, so dass alle gleichzeitig auf einer identischen Frequenz senden können, aber anhand des Codes eindeutig identifiziert werden können.
\textbf{Beispiel :} UMTS

\subsection{Störung}
\subsubsection{Abschattung}
Durch Hindernisse kann ein Signal abgeschattet (shadowing) werden. Der Empfang wird dadurch erschwert oder verhindert
\textbf{Beispiel :} Hochhäuser, Tunnel

\subsubsection{Reflexion}
Wellenförmiges Signal wird an grosser Fläche (spiegelartig) reflektiert. Reflektierte Signale sind schwächer als direkt empfangene (wenn sie denn überhaupt empfangen werden)

\subsubsection{Streuung}
Relativ kleine Hindernisse können das Signal streuen (scattering), das sich dann in verschiedene Richtungen weiterbewegt.

\subsubsection{Beugung}
Kanten können eine Beugung (diffraction)/Ablenkung des Signals verursachen.

\subsubsection{Dämpfung \& Verzögerung}
Empfangsleistung nimmt proportional zum Quadrat der Entfernung zwischen Sender und Empfänger ab(path loss). Zeitrahmensynchronisation ist ortsabhängig (time alginment)

\subsubsection{Interferenzen}
Konkurrieren mehrer Mobilstationen um das Übertragungsmedium können Interferenzen (Überlagerung zwischen getrennten Kanälen) auftreten.

\subsubsection{Mehrwegausbreitung-Multipath Fading}
\begin{itemize}
\item Aufgrund von Streuung, Beugung und REflexion kann das Signal das Ziel auf unterschiedlichen Wegen erreichen.
\item Der Impuls des Senders kann beim Empfänger mehrfach und zeitlich versetzt auftreten.
\item Die Stärke der eingehenden Signale schwankt, da die Ausbreitungswege unterschiedliche Dämpfung aufweisen können.
\item Unterschieden wird zwischen 
\begin{itemize}
\item Rayleigh Fading 
\item Time Dispersion
\end{itemize}
\end{itemize}